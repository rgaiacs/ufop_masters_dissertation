%!TEX root = ./main.tex

%% Title
\titlepage[Universidade Federal de Ouro Preto]{%
  Tese submetida ao Programa de Pós-Graduação em Ciência da Computação da Universidade Federal de Ouro Preto para obtenção do título de doutor em Ciência da Computação}

%% Resumo
\begin{abstract}[\smaller \thetitle \\ \vspace*{1cm} \smaller Resumo]
  %\thispagestyle{empty}
Como um dos maiores problemas é que... 
Este é um problema não resolvido pois...
Este trabalho `` \thetitle '' descreve ... 
Os resultados são que ...
\end{abstract}

\begin{abstract}[\smaller Title in English \\ \vspace*{1cm} \smaller Abstract]
  %\thispagestyle{empty}
One problem with $X$ is that... 
This is an unsolved problem because...
This work describes... 
Our results show that...
\end{abstract}


%% Declaração
\begin{abstract}[Declaração]
Esta tese é resultado de meu próprio trabalho, exceto onde referência explícita é feita ao trabalho de outros, e não foi submetida para obtenção de título nesta nem em outra universidade.
  \vspace*{1cm}
  \begin{flushright}
    Alan Robert Resende de Freitas
  \end{flushright}
\end{abstract}


%% Agradecimentos
\begin{abstract}[Agradecimentos]
Agradeço a \emph{meus pais}, por \dots.

Agradeço a \emph{minha família}, por \dots.

Agradeço a \dots.

Agradeço a \dots.

Agradeço a \dots.

Agradeço a \dots.

Agradeço a \dots.
\end{abstract}


%% Prefácio
\begin{abstract}[Prefácio]
%Resumo expandido... O prefácio é opcional. Aqui você pode explicar a que tipo de leitor se digire o texto.
\end{abstract}

%% ToC
\tableofcontents
% Lista de figuras
\listoffigures
% Insere lista de tabelas
\listoftables
% Insere lista de algoritmos (este comando está disponível se você utilizar o pacote algorithm2e)
% \listofalgorithms

%% Aqui você pode colocar uma citação antes de começar a introdução
% Esta citação é estritamente opcional
\frontquote{%
  Ser ou não ser.}%
  {William Shakespeare, 1564 -- 1616}
%% I don't want a page number on the following blank page either.
\thispagestyle{empty}
