%% Carrega pacotes de fontes especiais se quiser
%\usepackage{lmodern}
\usepackage{mathpazo}
%\usepackage{euler}
\usepackage[utf8]{inputenc}

%% Pacotes usuais
\usepackage{xspace}
\usepackage{tikz}
\usepackage{morefloats,subfig,afterpage}
\usepackage{mathrsfs} % script font
\usepackage{verbatim}


%% Usando o Babel para permitir outras línguas que serão usadas e misturadas facilmente
%\usepackage[ngerman,english]{babel}
\usepackage[english]{babel}

%% Alterações no sistema de citação
\usepackage{cite}
% \let\@OldCite\cite
% \renewcommand{\cite}[1]{\mbox{\!\!\!\@OldCite{#1}}}

%% Matemática
% A se fazer: refazer ou eliminar o maybemath
\usepackage{abmath}
\DeclareRobustCommand{\mymath}[1]{\ensuremath{\maybebmsf{#1}}}
% \DeclareRobustCommand{\parenths}[1]{\mymath{\left({#1}\right)}\xspace}
% \DeclareRobustCommand{\braces}[1]{\mymath{\left\{{#1}\right\}}\xspace}
% \DeclareRobustCommand{\angles}[1]{\mymath{\left\langle{#1}\right\rangle}\xspace}
% \DeclareRobustCommand{\sqbracs}[1]{\mymath{\left[{#1}\right]}\xspace}
% \DeclareRobustCommand{\mods}[1]{\mymath{\left\lvert{#1}\right\rvert}\xspace}
% \DeclareRobustCommand{\modsq}[1]{\mymath{\mods{#1}^2}\xspace}
% \DeclareRobustCommand{\dblmods}[1]{\mymath{\left\lVert{#1}\right\rVert}\xspace}
% \DeclareRobustCommand{\expOf}[1]{\mymath{\exp{\!\parenths{#1}}}\xspace}
% \DeclareRobustCommand{\eexp}[1]{\mymath{e^{#1}}\xspace}
% \DeclareRobustCommand{\plusquad}{\mymath{\oplus}\xspace}
% \DeclareRobustCommand{\logOf}[1]{\mymath{\log\!\parenths{#1}}\xspace}
% \DeclareRobustCommand{\lnOf}[1]{\mymath{\ln\!\parenths{#1}}\xspace}
% \DeclareRobustCommand{\ofOrder}[1]{\mymath{\mathcal{O}\parenths{#1}}\xspace}
% \DeclareRobustCommand{\SOgroup}[1]{\mymath{\mathup{SO}\parenths{#1}}\xspace}
% \DeclareRobustCommand{\SUgroup}[1]{\mymath{\mathup{SU}\parenths{#1}}\xspace}
% \DeclareRobustCommand{\Ugroup}[1]{\mymath{\mathup{U}\parenths{#1}}\xspace}
% \DeclareRobustCommand{\I}[1]{\mymath{\mathrm{i}}\xspace}
% \DeclareRobustCommand{\colvector}[1]{\mymath{\begin{pmatrix}#1\end{pmatrix}}\xspace}
\DeclareRobustCommand{\Rate}{\mymath{\Gamma}\xspace}
\DeclareRobustCommand{\RateOf}[1]{\mymath{\Gamma}\parenths{#1}\xspace}

%% Física
\usepackage{abhep}
\usepackage{hepnames}
\usepackage{hepunits}
\DeclareRobustCommand{\arXivCode}[1]{arXiv:#1}
\DeclareRobustCommand{\CP}{\ensuremath{\mathcal{CP}}\xspace}
\DeclareRobustCommand{\CPviolation}{\CP-violation\xspace}
\DeclareRobustCommand{\CPv}{\CPviolation}
\DeclareRobustCommand{\LHCb}{LHCb\xspace}
\DeclareRobustCommand{\LHC}{LHC\xspace}
\DeclareRobustCommand{\LEP}{LEP\xspace}
\DeclareRobustCommand{\CERN}{CERN\xspace}
\DeclareRobustCommand{\bphysics}{\Pbottom-physics\xspace}
\DeclareRobustCommand{\bhadron}{\Pbottom-hadron\xspace}
\DeclareRobustCommand{\Bmeson}{\PB-meson\xspace}
\DeclareRobustCommand{\bbaryon}{\Pbottom-baryon\xspace}
\DeclareRobustCommand{\Bdecay}{\PB-decay\xspace}
\DeclareRobustCommand{\bdecay}{\Pbottom-decay\xspace}
\DeclareRobustCommand{\BToKPi}{\HepProcess{ \PB \to \PK \Ppi }\xspace}
\DeclareRobustCommand{\BToPiPi}{\HepProcess{ \PB \to \Ppi \Ppi }\xspace}
\DeclareRobustCommand{\BToKK}{\HepProcess{ \PB \to \PK \PK }\xspace}
\DeclareRobustCommand{\BToRhoPi}{\HepProcess{ \PB \to \Prho \Ppi }\xspace}
\DeclareRobustCommand{\BToRhoRho}{\HepProcess{ \PB \to \Prho \Prho }\xspace}
\DeclareRobustCommand{\X}{\thesismath{X}\xspace}
\DeclareRobustCommand{\Xbar}{\thesismath{\overline{X}}\xspace}
\DeclareRobustCommand{\Xzero}{\HepGenParticle{X}{}{0}\xspace}
\DeclareRobustCommand{\Xzerobar}{\HepGenAntiParticle{X}{}{0}\xspace}
\DeclareRobustCommand{\epluseminus}{\Ppositron\!\Pelectron\xspace}
\DeclareRobustCommand{\protonproton}{\Pproton\APantiproton\xspace}


%% Você pode definir o espaçamento de linhas desta maneira
%\setallspacing{double}
%% ou uma seção por vez desta maneira
%\setfrontmatterspacing{double}

%% Define o título e autor da tese
\title{Um Modelo Baseado no Hepthesis para Monografias, Dissertações e Teses: Uma Adaptação para a ABNT}
\author{Alan Robert Resende de Freitas}

%% Metadados do PDF específicos do documento
\makeatletter
\@ifpackageloaded{hyperref}{%
\hypersetup{%
  pdftitle = {Um Modelo Baseado no Hepthesis para Monografias, Dissertações e Teses: Uma Adaptação para a ABNT},
  pdfsubject = {Alan Robert Resende de Freitas: Tese de Doutorado},
  pdfkeywords = {Inteligência Computacional, Sistemas Computacionais, Mineração de Dados},
  pdfauthor = {\textcopyright\ Alan Robert Resende de Freitas}
}}{}
\makeatother

