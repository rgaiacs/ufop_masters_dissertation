\chapter{Image Feature Extraction}

A subset of characteristics
with a high discriminative capacity
to be extracted
from the set of cutouts
was defined.
All selected features
refer to information related to the texture of the images
and do not undergo segmentation preprocessing.

\input{R/average-intensity.tex}

Maximum Intensity Value;

Minimum Intensity Value;

\section{Local Binary Pattern}

The Local Binary Pattern (LBP)
is a derivation of the Texture Unit
defined by \cite{wang_texture_1990}.
From the spatial domain,
we can define the LBP feature vector $l$
for each pixel $a$
by comparing the pixel to each of its 8 neighbors.
If the center pixel's value is greater than the neighbor's value,
we assign $0$ to that position in the vector.
Otherwise,
we write $1$.
For example,
consider that we have

\begin{center}
\begin{tabular}{|c|c|c|}
  \hline
  $a_{i - 1, j + 1} = 194$ & $a_{i, j + 1} = 144$ & $a_{i + 1, j + 1} = 132$ \\
  \hline
  $a_{i - 1, j} = 180$ & $a_{i, j} = 136$ & $a_{i + 1, j} = 136$ \\
  \hline
  $a_{i - 1, j - 1} = 167$ & $a_{i, j - 1} = 125$ & $a_{i + 1, j - 1} = 129$ \\
  \hline
\end{tabular}
\end{center}

The feature vector will be

\begin{center}
\begin{tabular}{|c|c|c|}
  \hline
  $l_{i - 1, j + 1} = 1$ & $l_{i, j + 1} = 1$ & $l_{i + 1, j + 1} = 0$ \\
  \hline
  $l_{i - 1, j} = 1$ & & $l_{i + 1, j} = 0$ \\
  \hline
  $l_{i - 1, j - 1} = 1$ & $l_{i, j - 1} = 0$ & $l_{i + 1, j - 1} = 0$ \\
  \hline
\end{tabular}
\end{center}

We represented as a matrix
that has $l_{i, j}$ empty.
We can store the matrix as a linear vector
by following the 8 neighbours in one direction.
For example,
if we start with $l_{i, j + 1}$
and go clockwise,
we will have

\begin{center}
\begin{tabular}{|c|}
  \hline $l_{i, j + 1} = 1$ \\
  \hline $l_{i + 1, j + 1} = 0$ \\
  \hline $l_{i + 1, j} = 0$ \\
  \hline $l_{i + 1, j - 1} = 0$ \\
  \hline $l_{i, j - 1} = 0$ \\
  \hline $l_{i - 1, j - 1} = 1$ \\
  \hline $l_{i - 1, j} = 1$ \\
  \hline $l_{i - 1, j + 1} = 1$ \\
  \hline
\end{tabular}
\end{center}

In total,
we have $2^8 = 256$ possible values for the LBP vector $l$.
The histogram of LBP in an image
should revel its texture information.

\input{R/local-binary-pattern.tex}

\section{Histogram of Oriented Gradients (HOG)}

\section{Gray-Level Co-Occurrence Matrix (GLCM);}

7 Hu Moments;

Haralick’s Texture Features (Haralick, 1979):

– Angular Second Moment (Energy);

– Contrast;

– Correlation;

– Variance;

– Entropy;

– Maximal Correlation Coefficient;

– Inverse Difference Moment (Homogeneity);

– Sum Average, Sum Variance and Sum Entropy;

– Difference Variance and Difference Entropy;

– Information Measure of Correlation I and II.

\input{R/range.tex}

\input{R/variance.tex}
