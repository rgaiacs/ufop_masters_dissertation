%% Estritamente opcional
%\begin{colophon}
%  Esta tese foi feita em \LaTeXe{} usando a classe ``hepthesis'' ~\cite{hepthesis}.
%\end{colophon}

% Aqui será definido o estilo de bibliografia utilizado
% Pacote para Padrão de Harvard
\bibliographystyle{abbrv}

%% No arquivo citacoes.bib ficam todas as nossas citações
% Apenas as citações que utilizamos em nosso texto aparecerão no fim do texto
% Para que as citações sejam compiladas, você precisa compilar o seu texto normalmente para que se saiba quais citações foram usadas. Depois, deve rodar o bibtex no seu texto para que um arquivo com as citações no formato abnt seja gerado. Depois, é necessário rodar o latex no arquivo principal novamente.

\bibliography{mythesis}

%% Várias pessoas preferem colocar estas tabelas aqui em vez de deixar a front matter enorme. 
%\listoffigures
%\listoftables

%% Aqui você pode incluir um índice remissivo. O índice remissivo organiza os termos importantes do seu texto por ordem alfabética. Para que um termo apareça no índice remissivo, ele precisa estar marcado com o comando \index no seu texto. Isso pode ser feito após a declaração das seções. O MakeIndex deve ser rodado no seu main.tex para gerar este indice.
%\printindex
